\documentclass[a4paper, 12pt]{article}
\usepackage[utf8]{inputenc}
\usepackage[czech]{babel}
\usepackage[left=2cm, top=3cm, text={17cm, 24cm}]{geometry}
\usepackage{graphicx}
\graphicspath{ {./images/} }
\usepackage{fancyhdr}
\usepackage{enumitem}
\usepackage{tabularx}
\usepackage{url}
\usepackage{xcolor}
\usepackage{listings}
\usepackage[unicode, colorlinks, hypertexnames=false, citecolor=red]{hyperref}
\usepackage{tabto}
\usepackage{placeins}
\hypersetup{colorlinks = true, hypertexnames = false}
\usepackage{pdflscape}
\usepackage{amsmath}
\usepackage{amssymb}
\usepackage{tikz}
\usetikzlibrary{automata, arrows.meta, positioning,shapes}
\usepackage{pdflscape}

\renewcommand{\labelenumi}{\alph{enumi})}

\setlength\parindent{0pt}

\usepackage{fancyvrb}

% [David Chocholatý]: pro definice, teorémy, ....
\usepackage{amsthm}
% - Definice.
\theoremstyle{definition}
\newtheorem{definition}{Definice}[section]

% - Teorém.
\theoremstyle{definition}
\newtheorem{theorem}{Teorém}[section]

% - Lemma.
\theoremstyle{definition}
\newtheorem{lemma}[theorem]{Lemma}

% - Příklad.
\theoremstyle{remark}
\newtheorem{example}{Příklad}[section]

\setlength{\parindent}{20pt}

% redefine \VerbatimInput
\RecustomVerbatimCommand{\VerbatimInput}{VerbatimInput}%
{fontsize=\footnotesize,
 %
 frame=lines,  % top and bottom rule only
 framesep=2em, % separation between frame and text
 rulecolor=\color{gray},
 %
 label=\fbox{\color{black}udpData.txt},
 labelposition=topline,
 %
 %commandchars=\|\(\), % escape character and argument delimiters for
                      % commands within the verbatim
 %commentchar=*        % comment character
}

\lstdefinestyle{BashInputStyle}{
  language=bash,
  basicstyle=\small\sffamily,
  %numbers=left,
  %numberstyle=\scriptsize,
  %numbersep=3pt,
  frame=tb,
  columns=fullflexible,
  backgroundcolor=\color{cyan!5},
  linewidth=0.9\linewidth,
  %xleftmargin=0.1\linewidth
}

\newcommand*{\Package}[1]{\texttt{#1}}

\definecolor{dkgreen}{rgb}{0,0.6,0}
\definecolor{gray}{rgb}{0.5,0.5,0.5}
\definecolor{mauve}{rgb}{0.58,0,0.82}

\lstdefinestyle{CLanguage}{
  language=C,
  aboveskip=3mm,
  belowskip=3mm,
  showstringspaces=false,
  columns=flexible,
  basicstyle={\small\ttfamily},
  numbers=none,
  numberstyle=\tiny\color{gray},
  keywordstyle=\color{blue},
  commentstyle=\color{dkgreen},
  stringstyle=\color{mauve},
  breaklines=true,
  breakatwhitespace=true,
  tabsize=3,
  inputencoding=utf8,
  extendedchars=true,
  literate={á}{{\'a}}1 {ú}{{\'a}}1 {é}{{\'e}}1 {í}{{\'i}}1   {ý}{{\'y}}1 {č}{{\v{c}}}1 {ť}{{\'t}}1 {ľ}{{\'l}}1 
}

\begin{document}

    %Titulni strana
    \begin{titlepage}
        \begin{center}
            \includegraphics[width=0.87\textwidth]{logo_cz.png}
            \vspace*{6cm}

            \Huge{\textbf{Teoretická informatika 2023/2024}}
            \vspace{0.5cm}
            
            \LARGE{Úkol 1}
            \vspace{1cm}
            
            \Large{David Chocholatý (xchoch09)}
            
           \vfill
		   \begin{flushright} 
		   Brno, \today
		   \end{flushright}
        \end{center}
    \end{titlepage}

\pagestyle{fancy}
\lhead{\bfseries Teoretická informatika 2023/2024 --- Úkol 1}
\rhead{\bfseries xchoch09}

\section{Příklad 1}
Mějme následující jazyky:

\begin{enumerate}[label=(\alph*)]
    \item $L_1 = \{ww^R \, | \, w \in \{a,b,c\}^\ast\}$
    \item $L_2 = \{w \, | \, w \in \{a,b,c\}^\ast \; \wedge \; \#_a(w) \, mod \, 2 = \#_b(w) \, mod \, 2 = 1\}$
\end{enumerate}

Rozhodněte a dokažte, zda jazyky $L_1 \cap L_2$ a $L_1 \cup L_2$ jsou regulární. Pro důkaz regularity sestrojte příslušný konečný automat, nebo gramatiku. Pro důkaz neregularity použijte Pumping lemma, nebo Myhill-Nerodovu větu. 

\subsection{Řešení}

Uvažujme zadané jazyky $L_1$ a $L_2$. Neformálně řečeno, jazyk $L_1$ představuje dobře známý bezkontextový jazyk, který obsahuje všechny řetězce nad $\Sigma = \{a,b,c\}$, které jsou konkatenací dvou stejně dlouhých řetězců takových, že druhý řetězec je reverzací prvního řetězce. Vyplývá, že $L_1 \notin \mathcal{L}_3$, přičemž by to bylo možné dokázat pomocí Pumping lemmatu podobně, jako je tomu následně pro $L_1 \cup L_2$.

Jazyk $L_2$ obsahuje všechny řetězce nad $\Sigma = \{a, b, c\}$ takové, že počet symbolů $a$ i $b$ je lichý. Jelikož počet kombinací sudosti a lichosti počtu symbolů $a$ a $b$ je konečný, lze řetězce jazyka $L_2$ přijímat například úplně definovaným deterministickým konečným automatem, a tudíž $L_1 \in \mathcal{L}_3$. Kompletní důkaz by bylo možné provést pomocí Myhill-Nerodovy věty nebo sestrojením zmiňovaného úplně definovaného deterministického konečného automatu.

\begin{itemize}
    \item $L_1 \cap L_2$

    Průnikem jazyků $L_1$ a $L_2$ vzniká nový jazyk 
    
    $$L_1 \cap L_2 = \{ww^R \, | \, w \in \{a,b,c\}^\ast \; \wedge \; \#_a(ww^R) \, mod \, 2 = \#_b(ww^R) \, mod \, 2 = 1\}.$$

    Jelikož jazyk $L_1 \cap L_2$ je průnikem regulárního a bezkontextového jazyka, lze z uzávěrových vlastností $\mathcal{L}_2$ vůči průniku s regulárními jazyky usoudit, že výsledný jazyk bude regulární (viz \cite{handbook}). To může pro nově vzniklý jazyk platit tehdy a jen tehdy, když
    
    $$L_1 \cap L_2 = \emptyset \in \mathcal{L}_3 \iff \nexists ww^R \in \{a, b, c\}^\ast: \#_a(ww^R) \, mod \, 2 = \#_b(ww^R) \, mod \, 2 = 1.$$

    \begin{lemma}
        \label{lemma1}
        $\nexists ww^R \in \{a, b, c\}^\ast: \#_a(ww^R) \, mod \, 2 = \#_b(ww^R) \, mod \, 2 = 1.$
    \end{lemma}

    \begin{proof}
        Důkaz bude proveden sporem. Předpokládejme, že
        
        $$\exists ww^R \in \{a, b, c\}^\ast: \#_a(ww^R) \, mod \, 2 = \#_b(ww^R) \, mod \, 2 = 1.$$

        Uvažujme libovolné $ww^R \in \{a,b,c\}^\ast$. Vyplývá, že $\#_a(ww^R) = 2 * \#_a(w)$ a $\#_b(ww^R) = 2 * \#_b(w)$. Nutně ovšem musí pro všechna $ww^R$ platit, že $\#_a(ww^R) \, mod \, 2 = \#_b(ww^R) \, mod \, 2 = 2$. Tudíž docházíme ke sporu s předpokladem, že existuje řetězec $ww^R$ obsahující lichý počet symbolů $a$ i $b$.
    \end{proof}

    Na základě lemmatu \ref{lemma1} platí, že $L_1 \cap L_2 = \emptyset \in \mathcal{L}_3$ (jedná se tedy o regulární jazyk), přičemž takový jazyk je možné přijmout úplně definovaným deterministickým konečným automatem

    $$M_1 = (\{s\}, \{a,b,c\}, \delta, s, \emptyset),$$

    kde

    \begin{align*}
        \delta: (s, a) &= \{s\}, \\
        (s, b) &= \{s\}, \\
        (s, c) &= \{s\}. \\            
    \end{align*}
    
    Stavový diagram úplně definovaného deterministického konečného automatu $M_1$ je vyobrazen na obrázku \ref{fig:m1}.

     \begin{figure}[ht!]
    \begin{center}
        \hspace*{0.15\linewidth}    
        
        \begin{tikzpicture}[shorten >=1pt,node distance=5cm,on grid,auto]
          \tikzstyle{every state}=[fill={rgb:black,1;white,10}]
        
          \node[state,initial,initial text=] (s_0)                 {$s$};          
        
          \path[->]
          (s_0) edge  [loop above]   node {$a, b, c$} ();
        \end{tikzpicture}
    \caption{Stavový diagram úplně definovaného deterministického konečného automatu $M_1$.}
    \label{fig:m1}
    \end{center}
    \end{figure}
    \FloatBarrier
    
    \item $L_1 \cup L_2$

    Sjednocením jazyků $L_1$ a $L_2$ vzniká nový jazyk

    $$L_1 \cup L_2 = \{ww^R \, | \, w \in \{a,b,c\}^\ast\} \cup \{w \, | \, w \in \{a,b,c\}^\ast \; \wedge \; \#_a(w) \, mod \, 2 = \#_b(w) \, mod \, 2 = 1\}.$$

    Na základě skutečnosti, že jazyk $L_1 \cup L_2$ je sjednocením regulárního a neregulárního (bezkontextového) jazyka, je předpokládáno, že $L_1 \cup L_2 \notin \mathcal{L}_3$.

    \begin{lemma}
        $L_1 \cup L_2 = \{ww^R \, | \, w \in \{a,b,c\}^\ast\} \cup \{w \, | \, w \in \{a,b,c\}^\ast \; \wedge \; \#_a(w) \, mod \, 2 = \#_b(w) \, mod \, 2 = 1\} \notin \mathcal{L}_3.$
    \end{lemma}

    \begin{proof}
        Důkaz sporem pomocí Pumping lemmatu. Předpokládejme, že $L_1 \cup L_2 \in \mathcal{L}_3$. Pak dle Pumping lemmatu platí, že

        \begin{align*}
            \exists k > 0: &\forall w \in \Sigma^\ast: w \in L_1 \cup L_2 \wedge |w| \geq k \implies \\
            &\exists x, y, z \in \Sigma^\ast: w = xyz \wedge y \neq \varepsilon \wedge |xy| \leq k \wedge \forall i \geq 0: xy^iz \in L_1 \cup L_2.
        \end{align*}

        Uvažujme libovolné $k > 0$ a zvolme $w = a^kbba^k$. Platí, že $w \in L_1 \cup L_2 \wedge |w| = 2 * k + 2 \geq k$. Pro každé rozdělení slova $w$ platí, že $w = xyz \wedge y \neq \varepsilon \wedge |xy| \leq k$. Zřejmě

        \begin{align*}
            x &= a^{\alpha_1}, 0 \leq \alpha_1 < k, \\
            y &= a^{\alpha_2}, 1 \leq \alpha_2 \leq k \wedge \alpha_1 + \alpha_2 = k, \\
            z &= a^{k - \alpha_1 - \alpha_2}bba^k. \\
        \end{align*}

        Zvolme $i = 0$. Pak $xy^0z = a^{\alpha_1}a^{k - \alpha_1 - \alpha_2}bba^k = a^{k - \alpha_2}bba^k$. Dle Pumping lemmatu by mělo platit, že $a^{k - \alpha_2}bba^k \in L_1 \cup L_2$. To platí tehdy a jen tehdy, když
        
        $$k - \alpha_2 = k \lor ((k - \alpha_2 + k) \, mod \, 2 = 2 \, mod \, 2 = 1).$$
        
        \begin{itemize}
            \item $k - \alpha_2 = k$:
            
            Nastane pouze v případě, že $\alpha_2 = 0$, ovšem to je ve sporu s předpokladem, že $\alpha_2 \geq 1$.
            \item $(k - \alpha_2 + k) \, mod \, 2 = 2 \, mod \, 2 = 1$:
            
            Nastane pouze v případě, že $(k - \alpha_2 + k) \, mod \, 2 = 1 \wedge 2 \, mod \, 2 = 1$. Druhá část výrazu ovšem nikdy nemůže nastat a vyplývá, že celý výraz neplatí.
        \end{itemize}
        Tudíž, $a^{k - \alpha_2}bba^k \notin L_1 \cup L_2$ a dochází ke sporu s předpokladem, že $L_1 \cup L_2 \in \mathcal{L}_3$.
    \end{proof}
\end{itemize}

\section{Příklad 2}
Uvažujme jazyk $L_3 = \{puvw \, | \, p, v \in \{a,b\}^\ast, u, w \in \{c,d\}^\ast, (p = v^R \lor u = w^R)\}$

\begin{enumerate}[label=(\alph*)]
    \item Sestrojte bezkontextovou gramatiku $G_3$ takovou, že $L(G_3) = L_3$.
    \item Sestrojte zásobníkový automat $Z_3$ takový, že $L(Z_3) = L_3$.
\end{enumerate}

\subsection{Řešení}

\begin{enumerate}[label=(\alph*)]
    \item Řešením je bezkontextová gramatika

    $$G_3 = (N, \Sigma, P, S),$$

    kde

    \begin{align*}
        N = \{&S, A, B, C, D\}, \\
        \Sigma = \{&a, b, c, d\}, \\
        P = \{&S \rightarrow AB \, | \, CD, \\
              &A \rightarrow B \, | \, aAa \, | \, bAb, \\
              &B \rightarrow \varepsilon \, | \, cB \, | \, dB, \\
              &C \rightarrow \varepsilon \, | \, aC \, | \, bC, \\
              &D \rightarrow C \, | \, cDc \, | \, dDd\}. \\            
    \end{align*}

    \item Řešením je zásobníkový automat $Z_3$, jehož stavový diagram je vyobrazen na obrázku \ref{fig:pushdownautomaton}. Notace pravidel je zapisována ve tvaru, kdy jako vrchol zásobníku je uvažován symbol nejvíce napravo.

    \begin{landscape}    
        \hfill
        \begin{figure}[ht!]
        \begin{center}
            \hspace*{0.15\linewidth}    
            
            \begin{tikzpicture}[shorten >=1pt,node distance=3cm,on grid,auto]
              \tikzstyle{every state}=[fill={rgb:black,1;white,10}]
            
              \node[state,initial,initial text=\#] (s_0)       at (0, 0)          {$q_0$};      \node[state]                    (s_1) at (3.5, 2) {$q_1$};              
          \node[state]                    (s_2) at (7, 2)  {$q_2$};          
          \node[state]                    (s_3) at (10.5, 2)  {$q_3$};
          \node[state]                    (s_4) at (14, 2)  {$q_4$};
          
          \node[state]                    (s_5) at (3.5, -2)  {$q_5$};
          \node[state]                    (s_6) at (7, -2)  {$q_6$};
          \node[state]                    (s_7) at (10.5, -2)  {$q_7$};
          \node[state]                    (s_8) at (14, -2)  {$q_8$};
          \node[state,accepting]                    (s_9) at (17.5, 0)  {$q_9$};
          %\node[state,accepting]                    (s_5) at (11, 0)  {$s_5$};
              %\node[state]                    (s_1) [right of=s_0]  {$p$};
              %\node[state]                    (s_2) [right of=s_1]  {$p$};
              %\node[state,accepting]          (s_3) [right of=s_2]  {$f$};              

              \path[->]
              (s_0) edge                node {$\varepsilon, \#/\#$}  (s_1)
              (s_1) edge                node [text width=1.0cm,align=left] {$\varepsilon, \#/\#$\\ $\varepsilon, a/a$\\ $\varepsilon, b/b$}  (s_2)
              (s_1) edge  [loop above]   node [text width=1.0cm,align=left] {$a, \#/\#a$\\ $a, a/aa$\\ $a, b/ba$\\ $b, \#/\#b$\\ $b, a/ab$\\ $b, b/bb$} ()
              (s_2) edge  [loop above]   node [text width=1.0cm,align=left] {$c, \#/\#$\\ $c, a/a$\\ $c, b/b$\\ $d, \#/\#$\\ $d, a/a$\\ $d, b/b$} ()
              (s_2) edge                node [text width=1.0cm,align=left] {$\varepsilon, \#/\#$\\ $\varepsilon, a/a$\\ $\varepsilon, b/b$}  (s_3)
              (s_3) edge                node {$\varepsilon, \#/\#$}  (s_4)
              (s_3) edge  [loop above]   node [text width=1.0cm,align=left] {$a, a/\varepsilon$\\ $b, b/\varepsilon$} ()
              (s_4) edge                node {$\varepsilon, \#/\varepsilon$}  (s_9)
              (s_4) edge  [loop above]   node [text width=1.0cm,align=left] {$c, \#/\#$\\ $d, \#/\#$} ()
              (s_0) edge                node {$\varepsilon, \#/\#$}  (s_5)
              (s_5) edge                node {$\varepsilon, \#/\#$}  (s_6)
              (s_5) edge  [loop below]   node [text width=1.0cm,align=left] {$a, \#/\#$\\ $b, \#/\#$} ()
              (s_6) edge                node [text width=1.0cm,align=left] {$\varepsilon, \#/\#$\\ $\varepsilon, c/c$\\ $\varepsilon, d/d$}  (s_7)
              (s_6) edge  [loop below]   node [text width=1.0cm,align=left] {$c, \#/\#c$\\ $c, c/cc$\\ $c, d/dc$\\ $d, \#/\#d$\\ $d, c/cd$\\ $d, d/dd$} ()
              (s_7) edge                node [text width=1.0cm,align=left] {$\varepsilon, \#/\#$\\ $\varepsilon, c/c$\\ $\varepsilon, d/d$}  (s_8)
              (s_7) edge  [loop below]   node [text width=1.0cm,align=left] {$a, \#/\#$\\ $a, c/c$\\ $a, d/d$\\ $b, \#/\#$\\ $b, c/c$\\ $b, d/d$} ()
              (s_8) edge                node {$\varepsilon, \#/\varepsilon$}  (s_9)
              (s_8) edge  [loop below]   node [text width=1.0cm,align=left] {$c, c/\varepsilon$\\ $d, d/\varepsilon$} ();
              %\path[->]
              %(s_0) edge                node {$a, S/Sa$}  (s_1)
              %(s_1) edge  [loop above]  node {$a, a/aa$}  ()
              %(s_1) edge                node {$b, a/\varepsilon$} (s_2)
              %(s_2) edge  [loop above]  node {$b, a/\varepsilon$} ()
              %(s_2) edge                node {$\varepsilon, S/\varepsilon$}  (s_3);
            \end{tikzpicture}
        \caption{Stavový diagram zásobníkového automatu $Z_3$.}
        \label{fig:pushdownautomaton}
        \end{center}
    \end{figure}                 
    \end{landscape}    
\end{enumerate}

\section{Příklad 3}
Rozhodněte a dokažte následující tvrzení:

\begin{enumerate}[label=(\alph*)]
    \item $\exists L_1 \in \mathcal{L}_2 \, \backslash \, \mathcal{L}_3$ takový, že jeho doplněk $\overline{L_1}$ je konečný jazyk.
    \item $\exists L_1 \in \mathcal{L}_2 \, \backslash \, \mathcal{L}_3$ takový, že $\forall L_2 \in \mathcal{L}_3: L_1 \cap L_2 \in \mathcal{L}_2 \, \backslash \, \mathcal{L}_3$
    \item $\exists L_1 \in \mathcal{L}_3$ takový, že $\forall L_2 \in \mathcal{L}_2 \, \backslash \, \mathcal{L}_3: L_1 \cap L_2 \in \mathcal{L}_2 \, \backslash \, \mathcal{L}_3$
\end{enumerate}

\subsection{Řešení}

\begin{enumerate}[label=(\alph*)]
    \item Tvrzení neplatí. Uvažujme libovolný jazyk $L_1 \in \mathcal{L}_2 \, \backslash \, \mathcal{L}_3$. Zároveň takový jazyk není konečný, jelikož by náležel do třídy konečných jazyků, značené $\mathcal{L}_{FIN}$, $\mathcal{L}_{FIN} \subset \mathcal{L}_3$.
    
    Dále bude proveden důkaz sporem. Stanovme, že $\overline{L_1} \in \mathcal{L}_{FIN}$ a předpokládejme, že $L_1 \in \mathcal{L}_2 \, \backslash \, \mathcal{L}_3$. Jelikož $\overline{L_1} \in \mathcal{L}_{FIN}$, potom platí, že $\overline{L_1} \in \mathcal{L}_3$. Dle uzavřenosti třídy regulárních jazyků vůči operaci doplňku vyplývá, že $L_1 \in \mathcal{L}_3$, přičemž dochází ke sporu.

    \item Tvrzení neplatí. Důkaz bude proveden sporem. Uvažujme libovolný jazyk $L_1 \in \mathcal{L}_2 \, \backslash \, \mathcal{L}_3$. Dále stanovme, že $L_2 = \emptyset$, přičemž $L_2 \in \mathcal{L}_3$. Potom nutně platí, že $L_1 \cap L_2 = \emptyset \in \mathcal{L}_3$ a dochází ke sporu.
    \item Tvrzení platí. Stanovme, že $L_1 = \Sigma^\ast \in \mathcal{L}_3$. Dále uvažujme libovolný jazyk $L_2 \in \mathcal{L}_2 \, \backslash \, \mathcal{L}_3$, přičemž $L_2 \subset L_1$. Vyplývá, že $L_1 \cap L_2 = L_2 \in \mathcal{L}_2 \, \backslash \, \mathcal{L}_3$.
\end{enumerate}

\section{Příklad 4}
Uvažujme jazyk $L = \{w \in \{a,b\}^\ast \, | \, \#_a(w) \geq 2 \; \lor \; \#_b(w) = 0\}$. Sestrojte relaci pravé kongruence $\sim$, která splňuje následující dvě podmínky: 1) $L$ je sjednocením některých tříd rozkladu $\Sigma^\ast / \sim$ a 2) index $\sim$ je o jedna větší než index $\sim_L$.

\subsection{Řešení}
Nejprve sestrojíme úplně definovaný deterministický konečný automat, který neobsahuje nedostupné stavy. Pro to, aby index $\sim$ byl o jedna větší než index $\sim_L$, je nejprve zapotřebí sestrojit minimální úplně definovaný deterministický konečný automat, abychom zjistili počet tříd rozkladu $\Sigma^\ast / \sim_L$.

Minimální úplně definovaný deterministický konečný automat pro jazyk $L$ je vyobrazen na obrázku \ref{fig:minimal}. Důkaz toho, že se jedná o minimální úplně definovaný deterministický konečný automat, by bylo možné provést s využitím Myhill-Nerodovy věty nebo algoritmu pro převod úplně definovaného deterministického konečného automatu na redukovaný deterministický konečný automat.


\begin{figure}[ht!]
    \begin{center}
        \hspace*{0.15\linewidth}    
        
        \begin{tikzpicture}[shorten >=1pt,node distance=5cm,on grid,auto]
          \tikzstyle{every state}=[fill={rgb:black,1;white,10}]
        
          \node[state,initial,accepting,initial text=] (s_0)       at (0, 0)          {$s_0$};
          \node[state,accepting]                    (s_1) at (3, 2) {$s_1$};
          \node[state,accepting]                    (s_2) [right of=s_1]  {$s_2$};
          \node[state]                    (s_3) at (3, -2)  {$s_3$};
          \node[state]                    (s_4) [right of=s_3]  {$s_4$};
          %\node[state]                    (s_5) [below of=s_2]  {$s_3$};
          %\node[state]                    (s_2) at (1, -1.5)  {$t$};
        
          \path[->]
          (s_0) edge                node {$a$}  (s_1)
          (s_1) edge                node {$a$}  (s_2)
          (s_1) edge                node {$b$}  (s_4)
          (s_0) edge                node {$b$}  (s_3)
          (s_3) edge                node {$a$}  (s_4)
          (s_4) edge                node {$a$}  (s_2)
          (s_3) edge  [loop below]   node {$b$} ()
          (s_4) edge  [loop below]   node {$b$} ()
          (s_2) edge  [loop above]   node {$a, b$} ();
          %(s_0) edge  [loop above]   node {$a, b, c$} ();

          %\path[->]
          %(s_0) edge                node {a}  (s_1)
          %(s_1) edge                node {b}  (s_2)
          %(s_2) edge                node {c}  (s_0);
        \end{tikzpicture}
    \caption{Stavový diagram minimálního úplně definovaného deterministického konečného automatu pro jazyk $L$.}
    \label{fig:minimal}
    \end{center}
    \end{figure}
    \FloatBarrier

Úplně definovaný deterministický konečný automat pro relaci pravé kongruence $\sim$ nesmí být minimální. Tento konečný automat je vyobrazen na obrázku \ref{fig:notminimal}. Dále rozklad $\Sigma^\ast / \sim$ musí obsahovat navíc právě jednu třídu oproti rozkladu $\Sigma^\ast / \sim_L$. Tudíž, index relace $\sim$ bude o jedna větší oproti indexu relace $\sim_L$.

\begin{figure}[ht!]
    \begin{center}
        \hspace*{0.15\linewidth}    
        
        \begin{tikzpicture}[shorten >=1pt,node distance=5cm,on grid,auto]
          \tikzstyle{every state}=[fill={rgb:black,1;white,10}]
        
          \node[state,initial,accepting,initial text=] (s_0)       at (0, 0)          {$s_0$};
          \node[state,accepting]                    (s_1) at (3, 2) {$s_1$};
          \node[state,accepting]                    (s_2) [right of=s_1]  {$s_2$};
          \node[state]                    (s_3) at (3, -2)  {$s_3$};
          \node[state]                    (s_4) [right of=s_3]  {$s_4$};
          \node[state,accepting]                    (s_5) at (11, 0)  {$s_5$};
          %\node[state]                    (s_5) [below of=s_2]  {$s_3$};
          %\node[state]                    (s_2) at (1, -1.5)  {$t$};
        
          \path[->]
          (s_0) edge                node {$a$}  (s_1)
          (s_1) edge                node {$a$}  (s_2)
          (s_1) edge                node {$b$}  (s_4)
          (s_0) edge                node {$b$}  (s_3)
          (s_3) edge                node {$a$}  (s_4)          
          (s_3) edge  [loop below]   node {$b$} ()
          (s_4) edge  [loop below]   node {$b$} ()
          (s_2) edge  [loop above]   node {$a$} ()
          (s_2) edge                node {$b$}  (s_5)
          (s_4) edge                node {$a$}  (s_5)
          (s_5) edge  [loop right]   node {$a, b$} ();
          %(s_0) edge  [loop above]   node {$a, b, c$} ();

          %\path[->]
          %(s_0) edge                node {a}  (s_1)
          %(s_1) edge                node {b}  (s_2)
          %(s_2) edge                node {c}  (s_0);
        \end{tikzpicture}
    \caption{Stavový diagram neminimálního úplně definovaného deterministického konečného automatu pro jazyk $L$.}
    \label{fig:notminimal}
    \end{center}
    \end{figure}
    \FloatBarrier

\newpage
\begin{definition}
    \label{langdef}
    Nechť $M = (Q, \Sigma, \delta, s_0, F)$ je nedeterministický konečný automat a nechť $q \in Q$. Jazyk přístupových řetězců stavu $q$ je definován následovně

    $$L^{-1}(q) = \{w \in \Sigma^\ast \; | \; (s_0, w) \vdash^\ast (q, \varepsilon)\}.$$
\end{definition}

Dále za použití definice \ref{langdef} najdeme jazyky přístupových řetězců nad $\Sigma = \{a, b\}$ všech stavů neminimálního úplně definovaného deterministického konečného automatu:

\begin{align*}
    L^{-1}(s_0) &= \{w \in \Sigma^\ast \; | \; \#_a(w) = 0 \wedge \#_b(w) = 0\}, \\
    L^{-1}(s_1) &= \{w \in \Sigma^\ast \; | \; \#_a(w) = 1 \wedge \#_b(w) = 0\}, \\
    L^{-1}(s_2) &= \{w \in \Sigma^\ast \; | \; \#_a(w) \geq 2 \wedge \#_b(w) = 0\}, \\
    L^{-1}(s_3) &= \{w \in \Sigma^\ast \; | \; \#_a(w) = 0 \wedge \#_b(w) \geq 1\}, \\
    L^{-1}(s_4) &= \{w \in \Sigma^\ast \; | \; \#_a(w) = 1 \wedge \#_b(w) \geq 1\}, \\    
    L^{-1}(s_5) &= \{w \in \Sigma^\ast \; | \; \#_a(w) \geq 2 \wedge \#_b(w) \geq 1\}. \\
\end{align*}

Najdeme rozklad množiny $\Sigma^\ast$, který je daný jako množina jazyků přístupových řetězců jednotlivých stavů:

$$P = \{L^{-1}(s_0), L^{-1}(s_1), L^{-1}(s_2), L^{-1}(s_3), L^{-1}(s_4), L^{-1}(s_5)\}.$$

Najdeme pravou kongruenci k tomuto rozkladu takovou, že $P = \Sigma^\ast / \sim$:

\begin{align*}
    \forall u,v \in \Sigma^\ast: u \sim v \iff &(\#_a(u) = 0 \wedge \#_a(v) = 0 \wedge \#_b(u) = 0 \wedge \#_b(v) = 0) \lor \\
    &(\#_a(u) = 1 \wedge \#_a(v) = 1 \wedge \#_b(u) = 0 \wedge \#_b(v) = 0) \lor \\
    &(\#_a(u) \geq 2 \wedge \#_a(v) \geq 2 \wedge \#_b(u) = 0 \wedge \#_b(v) = 0) \lor \\
    &(\#_a(u) = 0 \wedge \#_a(v) = 0 \wedge \#_b(u) \geq 1 \wedge \#_b(v) \geq 1) \lor \\
    &(\#_a(u) = 1 \wedge \#_a(v) = 1 \wedge \#_b(u) \geq 1 \wedge \#_b(v) \geq 1) \lor \\    
    &(\#_a(u) \geq 2 \wedge \#_a(v) \geq 2 \wedge \#_b(u) \geq 1 \wedge \#_b(v) \geq 1). \\    
\end{align*}

\begin{itemize}
    \item Jistě jde o relaci ekvivalence, protože vznikla z rozkladu množiny.
    \item Jistě jde o pravou kongruenci, protože jde o ekvivalenci, které vznikla pomocí jazyka přístupových řetězců stavů konečného automatu.
    \item Sjednocením některých tříd rozkladu $\Sigma^\ast / \sim$ získáme jazyk $L$:

    $$L = L^{-1}(s_0) \cup L^{-1}(s_1) \cup L^{-1}(s_2) \cup L^{-1}(s_5) = \{w \in \{a, b\}^\ast \; | \; \#_a(w) \geq 2 \lor \#_b(w) = 0\}.$$

    Index $\sim$ je o jedna větší oproti indexu prefixové ekvivalence $\sim_L$, neboť $|\Sigma^\ast / \sim| = 6$ a $|\Sigma^\ast / \sim_L| = 5$.
\end{itemize}

\newpage

\bibliographystyle{czechiso}
\renewcommand{\refname}{Použitá literatura}
\bibliography{main}

\end{document}
